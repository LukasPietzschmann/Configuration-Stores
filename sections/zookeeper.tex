\subsection{Overview}
\begin{frame}
	\frametitle{History}
	\begin{enumerate}
		\item December 2006 - first commit
		\item November 2007 - 0.0.1 on Sourceforge
		\item June 2008 - moved to Apache
		\item June 2010 - (Best paper award) Erstes Paper https://www.usenix.org/legacy/event/usenix10/tech/full_papers/Hunt.pdf
		\item November 2010 - ZooKeeper moves to top level project
		\item Januar 2023 - release-3.8.1
	\end{enumerate}
\end{frame}

\begin{frame}
	\frametitle{Popularity}
	\framesubtitle{According to Google}
	\centering
	\pgfplotsset{width=\textwidth, height=.7\textheight, compat=1.18}
	\pgfplotstableread[col sep=comma,]{../images/Zookeeperpopularitygoogle.csv}\datatable
	\tikzstyle{every pin}=[
	rounded corners,
	fill=white,
	draw=black,
	font=\footnotesize,
	]
	\tikzstyle{small dot}=[fill=black,circle,scale=0.3]
	\href{https://trends.google.com/trends/explore?cat=5&date=all&q=Zookeeper}{
		\begin{tikzpicture}
			\begin{axis}[
					axis background/.style={fill=accent!3},
					xlabel=Year,
					ylabel=Search Queries,
					axis line style={rounded corners},
					ytick=\empty,
					xtick={0, 234},
					xticklabels={01/2004, 06/2023},
					x tick label style = {font=\footnotesize, align=center},
					extra x ticks={90, 163.15},
					extra x tick labels={09/2011, 08/2017},
					extra x tick style={grid=major},
				]
				\addplot[color=accent, smooth] table [x=a, y=b, col sep=comma, x expr=\coordindex]{\datatable};
				\node [coordinate, pin=150:{Peak}, small dot] at (axis cs:163.15,81) {};
			\end{axis}
		\end{tikzpicture}}
\end{frame}

\begin{frame}
	\frametitle{What is zookeeper?}
	\framesubtitle{Zookeeper is \ldots}
	\textcolor<2>{gray}{\ldots~a (1)~highly available, (2)~scalable,
		(3)~\textcolor<2>{accent}{distributed}, (4)~\textcolor<2>{accent}{configuration},
		(5)~consensus, (6)~group membership, (7)~leader election, (8)~naming, and
		(9)~coordination \textcolor<2>{accent}{service}}
\end{frame}

\begin{frame}
	\frametitle{What is zookeeper again?}
	\begin{columns}[c]
		\begin{column}{0.45\textwidth}
			\begin{block}[Distributed]
				\begin{itemize}
					\item Fault-tolerant architecture
					\item Ordered updates and strong persistence guarantees
					\item Quorum-based replication
					\item Leader election
				\end{itemize}
			\end{block}
		\end{column}
		\begin{column}{0.45\textwidth}
			\begin{block}[Configuration store]
				\begin{itemize}
					\item Filesystem-like data model
					\item Watches for data changes
					\item Access control
				\end{itemize}
			\end{block}
		\end{column}
	\end{columns}
\end{frame}

\subsection{Architecture}
\begin{frame}
	\contourlength{.09em}
	\frametitle{Architecture}
	\centering
	\begin{tikzpicture}
		\begin{scope}
			\node [roundednode] (S1) [] {Server 1};
			\node [roundednode] (S2) [right=of S1] {Server 2};
			\node [roundednode, draw=accent, fill=accent!3] (S3) [right=of S2] {Server 3};
			\node [roundednode] (S4) [right=of S3] {Server 4};
			\node [roundednode] (S5) [right=of S4] {Server 5};
			\node [draw=none] (L) [above=of S2, yshift=1cm] {Leader};
			\draw [textarrow] (L) to[bend left=20] (S3);
			\node [draw=none] (ZS) [above=of S3] {\contour*{white}{\textcolor{black}{Zookeeper
						Service}}};
			\node [roundednode, node on layer=background, dashed, draw=darkgray, fill=gray!10, fit=(S1)(S2)(S3)(S4)(S5)(ZS)] {};

			\draw [arrow] (S1) to [bend left=20] (S3);
			\draw [arrow] (S2) to (S3);
			\draw [arrow] (S4) to (S3);
			\draw [arrow] (S5) to [bend right=20] (S3);

			\node [roundednode] (C4) [below=of S3, yshift=-.5cm] {Client 4};
			\node [roundednode] (C5) [right=of C4] {Client 5};
			\node [roundednode] (C6) [right=of C5] {Client 6};
			\node [roundednode] (C7) [right=of C6] {Client 7};
			\node [roundednode] (C3) [left=of C4] {Client 3};
			\node [roundednode] (C2) [left=of C3] {Client 2};
			\node [roundednode] (C1) [left=of C2] {Client 1};

			\draw [arrow] (C1) to (S1);
			\draw [arrow] (C2) to (S1);
			\draw [arrow] (C3) to (S2);
			\draw [arrow] (C4) to (S2);
			\draw [arrow] (C5) to (S4);
			\draw [arrow] (C6) to (S5);
			\draw [arrow] (C7) to (S5);
		\end{scope}
	\end{tikzpicture}
\end{frame}

\subsection{Data model}

\subsection{Operations}
