\documentclass[aspectratio=169, usepdftitle=false]{beamer}

\usepackage{lipsum}

\usepackage[english, secslide, subsecslide]{awesome-beamer/beamerthemeawesome}

\setminted{
	numbers=none,
	frame=none
}

\title[Distributed Config Stores]{Highly Available Distributed Configuration Stores}
\author[Kai, Tim, Lukas]{Kai Anter, Tim Heibel, Lukas Pietzschmann}
\subtitle{}
\email{}
\institute{Institute of Institute of Distributed Systems}
\uni{University of Ulm}
\location{Ulm}
\date{07/26/2023}
\background{background.jpg}

\begin{document}

\maketitle

\section{Section 1}
\begin{frame}[fragile]
	\frametitle{Boxes}
	\begin{columns}[t]
		\begin{column}{0.45\textwidth}
			\begin{block}[Some Title]
				This is an important information
			\end{block}
			\begin{alertblock}[Argh]
				Gotcha!
			\end{alertblock}
		\end{column}
		\begin{column}{0.45\textwidth}
			\begin{examples}[Example 1]
				\begin{verbatim}
\begin{alertblock}[Argh]
	Gotcha!
\end{alertblock}
				\end{verbatim}
			\end{examples}
			\begin{definition}
				I'm a definition with no title
			\end{definition}
		\end{column}
	\end{columns}
\end{frame}
\begin{frame}
	\frametitle[Another Title]{Example Slide}
	\framesubtitle{With a subtitle}
	\begin{columns}[t]
		\begin{column}{0.45\textwidth}
			\lipsum[1][1-5]
		\end{column}
		\begin{column}{0.45\textwidth}
			\textbf{Items:}
			\begin{itemize}
				\item Item 1
				\item Item 2
				\item Item 3
				\item Item 4
				\item Item 5
			\end{itemize}
		\end{column}
	\end{columns}
\end{frame}

\tikzstyle{every picture}+=[remember picture]
\tikzstyle{na} = [shape=rectangle,inner sep=0pt,text depth=0pt]
\begin{frame}
	\frametitle{Another Slide}
	\begin{itemize}
		\item That info is sooooooo \tikz\node[na] (A) {\uline{important}};
		\item This is a key \tikz\node[na] (D) {\uline{word}};, isn't it?
	\end{itemize}
	\vskip2em
	\uncover<2->{
		\Rightarrow~Two important \tikz \node[na] (B) {\uline{items}};!
		\begin{tikzpicture}[overlay, remember picture]
			\draw[textarrow] (A) to [bend left=40] (B);
			\draw[textarrow] (D) to (B);
		\end{tikzpicture}
	}
\end{frame}

\section{Section 2}
\subsection{Subsection 1}
\begin{frame}
	\frametitle{Cool picture}
	\framesubtitle{With animations}
	\begin{columns}
		\begin{column}{0.65\textwidth}
			\begin{tikzpicture}[
					remember picture,
					overlay,
				]
				\node[squarenode](P1)[background default draw=black, draw=accent, draw on=<3>, yshift=1cm]{P1};
				\node[squarenode](P2)[right=of P1, background default draw=black, draw=accent, draw on=<3>]{P2};
				\node[squarenode](P3)[right=of P2, background default draw=black, draw=accent, draw on=<3>]{P3};
				\node[squarenode](P4)[right=of P3, background default draw=black, draw=accent, draw on=<3>]{P4};
				\node[squarenode](P5)[right=of P4, background default draw=black, draw=accent, draw on=<3>]{P5};
				\node[draw=none](etc)[right=of P5]{...};
				\node[squarenode, draw=accent!70, background default draw=black, draw on=<{2,6,7}>, fit=(P1) (P2) (P3) (P4) (P5) (etc)] {};
				\node[draw=none] at (6.2, 1.5) {\fontspec{Symbola}\symbol{"1F512}};

				\node[squarenode](N1)[below=of P2, background default draw=black, draw=accent, draw on=<4>, visible on=<{4-7}>]{N1};
				\node[squarenode](N2)[right=of N1, background default draw=black, draw=accent, draw on=<4>, visible on=<{4-7}>]{N2};
				\node[squarenode](N3)[right=of N2, background default draw=black, draw=accent, draw on=<4>, visible on=<{4-7}>]{N3};

				\node[squarenode](P1p)[below=1.5cm of P1, visible on=<{7-9}>]{P1'};
				\node[squarenode](P2p)[right=of P1p, visible on=<{7-9}>]{P2'};
				\node[squarenode](P3p)[right=of P2p, visible on=<{7-9}>]{P3'};
				\node[squarenode](P4p)[right=of P3p, visible on=<{7-9}>]{P4'};
				\node[squarenode](P5p)[right=of P4p, visible on=<{7-9}>]{P5'};
				\node[draw=none, visible on=<{7-9}>](etcp)[right=of P5p]{...};
				\node[squarenode, draw=green, background default draw=black, visible on=<{7-9}>, draw on=<7>, fit=(P1p) (P2p) (P3p) (P4p) (P5p) (etcp)] {};
				\node[draw=none, visible on=<{7-9}>] at (6.55, -0.55) {\fontspec{Symbola}\symbol{"1F512}};

				\node[squarenode](S)[below=of N2, background default draw=black, draw=blue, draw on=<6>, visible on=<6>]{S};

				\draw[arrow, visible on=<{4-7}>] (P1) -- (N1);
				\draw[arrow, visible on=<{4-7}>] (P2) -- (N2);
				\draw[arrow, visible on=<{4-7}>] (P3) -- (N1);
				\draw[arrow, visible on=<{4-7}>] (P4) -- (N2);
				\draw[arrow, visible on=<{4-7}>] (P5) -- (N3);

				\draw[arrow, visible on=<7>] (N1) -- (P1p);
				\draw[arrow, visible on=<7>] (N2) -- (P2p);
				\draw[arrow, visible on=<7>] (N1) -- (P3p);
				\draw[arrow, visible on=<7>] (N2) -- (P4p);
				\draw[arrow, visible on=<7>] (N3) -- (P5p);

				\draw[arrow, visible on=<6>] (N1) -- (S);
				\draw[arrow, visible on=<6>] (N2) -- (S);
				\draw[arrow, visible on=<6>] (N3) -- (S);

				\draw[arrow, dashed, visible on=<8>] (P1) to [bend left=50] (P1p);
				\draw[arrow, dashed, visible on=<8>] (P2) to [bend left=20] (P2p);
				\draw[arrow, dashed, visible on=<8>] (P3) to [bend left=0] (P3p);
				\draw[arrow, dashed, visible on=<8>] (P4) to [bend right=20] (P4p);
				\draw[arrow, dashed, visible on=<8>] (P5) to [bend right=50] (P5p);

				\draw[arrow, dashed, visible on=<9>] (P1) to [bend left=50] (P1p);
				\draw[arrow, dashed, visible on=<9>] (P1) to (P3p);
				\draw[arrow, dashed, visible on=<9>] (P1) to (P4p);
				\draw[arrow, dashed, visible on=<9>] (P2) to [bend left=20] (P1p);
				\draw[arrow, dashed, visible on=<9>] (P2) to (P5p);
				\draw[arrow, dashed, visible on=<9>] (P3) to (P2p);
				\draw[arrow, dashed, visible on=<9>] (P4) to (P4p);
				\draw[arrow, dashed, visible on=<9>] (P4) to (P3p);
				\draw[arrow, dashed, visible on=<9>] (P5) to [bend right=50] (P5p);
				\draw[arrow, dashed, visible on=<9>] (P5) to [bend right=30] (P4p);
			\end{tikzpicture}
		\end{column}
		\begin{column}{0.25\textwidth}
			Wow\only<2->{, how cool}
		\end{column}
	\end{columns}
\end{frame}

\begin{frame}[fragile]
	\frametitle{Code}
	Here's some code in a box:
	\begin{block}
		\begin{minted}{c}
#define true false
int main() {
	if (true == false)
		printf("I'm in a parallel universe!\n");
	return 0;
}
		\end{minted}
	\end{block}
\end{frame}

\begin{frame}
	\frametitle{No information shown}
	\lipsum[2][1-8]
\end{frame}

\end{document}