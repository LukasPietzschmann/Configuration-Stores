\documentclass[aspectratio=169, usepdftitle=false]{beamer}

\makeatletter
\appto\input@path{{awesome-beamer}, {smile}}
\makeatother

\usepackage[english, secslide, subsecslide]{beamerthemeawesome}
\usepackage{pgfplots}
\usepackage{pgfplotstable}
\usepackage{microtype}
\usepackage{tikzducks}
\usepackage{soul}
\usepackage{pgf-umlsd}
\usepackage{tikzlings-sheep}

\usetikzlibrary{trees}

\usetikzlibrary{graphs, graphdrawing}
\usetikzlibrary{mindmap}
\usegdlibrary{layered}
\usegdlibrary{force}

\usepackage[
	duration=30,
	enduserslide=26,
	hidenotes
]{pdfpc}

\addbibresource{refs.bib}

\setminted{
	numbers=none,
	frame=none
}
\usemintedstyle{bw}

\title[Highly Available Distributed Config Stores]{Highly Available Distributed Configuration Stores}
\author[Kai, Tim, Lukas]{Kai Anter, Tim Heibel, Lukas Pietzschmann}
\subtitle{}
\email{}
\institute{Institute of Distributed Systems}
\uni{University of Ulm}
\location{Ulm}
\date{07/26/2023}
\background{background.jpg}

\begin{document}

\maketitle

\section{Motivation}
\input{sections/motivation.tex}

\section{Consul}
% Consul Architecture
% Pretty Good: http://man.hubwiz.com/docset/Consul.docset/Contents/Resources/Documents/docs/internals/architecture.html
% [Consul Architecture](https://developer.hashicorp.com/consul/docs/architecture)
% [Consul Vocabulary](https://developer.hashicorp.com/consul/docs/install/glossary)
% [Consensus Protocol](https://developer.hashicorp.com/consul/docs/architecture/consensus)
% **Fokus: [KVStore](https://developer.hashicorp.com/consul/docs/dynamic-app-config/kv)**
% - 

\subsection{Overview}

\begin{frame}
	\frametitle{General}
	\begin{itemize}
		\item Developed by \emph{Hashicorp}
		\item Released in 2014 as a Service Discovery Platform
		\item Implemented in \emph{Go}
		\item Now: Service-To-Service encryption, Health Checks, KV Store, \ldots
		\item Leader-based replication with Raft
		      % DNS to return healthy services
	\end{itemize}
\end{frame}

\begin{frame}
	\frametitle{Communication Types}
	% Two Types: Raft for Consensus, Serf for Gossip
	\begin{columns}[c]
		\begin{column}{0.5\textwidth}
			\begin{block}[Raft Protocol (Consensus)]
				\begin{itemize}
					\item<2-> Cluster State Replication
					\item<3-> Example: KV store, service and node IP addresses, configuration
					\item<4-> Crash Tolerance: (N/2) + 1
						\begin{itemize}
							\item 3 Nodes: 1 Crash
							\item 5 Nodes: 2 Crashes
							\item 7 Nodes: 3 Crashes
						\end{itemize}
				\end{itemize}
			\end{block}
		\end{column}
		\begin{column}{0.5\textwidth}
			\begin{block}[Surf Protocol (Gossip)]
				\begin{itemize}
					\item<5-> Perform and distribute \emph{service health checks}
					\item<6-> Examples for health checks: via HTTP GET Request, gRPC, TCP, UDP
						% https://developer.hashicorp.com/consul/docs/services/usage/checks
				\end{itemize}
			\end{block}
		\end{column}
	\end{columns}
\end{frame}

\begin{frame}
	\frametitle{Architecture}
	\contourlength{.09em}
	\begin{tikzpicture}[overlay]
		\node[roundednode, xshift=-1.9cm,yshift=-1.5cm, text width=3cm](logo) at
		(current page.north east) {\small
		\textcolor{yellow}{---} Raft consensus\\
		\textcolor{red}{---} KV request flow\\
		\textcolor{violet}{---} LAN gossip\\
		\textcolor{green}{---} WAN gossip\\};
	\end{tikzpicture}
	\centering
	\begin{tikzpicture}
		\begin{scope}
			\node [roundednode] (S1) [] {Server 1};
			\node [roundednode, draw=accent, fill=accent!10] (S2) [right=of S1] {Server 2};
			\node [roundednode] (S3) [right=of S2] {Server 3};

			\node [draw=none, visible on=<2>] (L) [above=of S1, yshift=1cm] {Leader};
			\draw [textarrow, visible on=<2>] (L) to[bend left=20] (S2);

			\foreach \i in {1,...,3}{
				\node [roundednode] (C\i) [below=of S\i, yshift=-5mm] {Client \i};
			}

			\node [draw=none] (DS1) [above=of S2] {\contour{white}{Datacenter A}};
			\node [roundednode, node on layer=background,draw=darkgray, dashed,
			fill=gray!10, inner sep=3mm] [fit=(S3)(C1)(DS1)] {};

			\node [roundednode] (APP) [below=of C3, yshift=-5mm] {Application};

			\begin{scope}[visible on=<6->]
				\node [roundednode] (SN) [left=of S1, xshift=-1cm] {Server n};
				\node [draw=none] (PC) [left=of C1, xshift=-1cm] {\phantom{C}};
				\node [draw=none] (DS2) [above=of SN] {Datacenter B};
				\node [roundednode, visible on=<6->, node on
				layer=background,draw=darkgray, dashed, fill=gray!10, inner sep=2mm] [fit=(SN)(PC)(DS2)] {};
			\end{scope}

			\foreach \i in {1,...,3}{
				\foreach \j in {1,...,3}{
					\draw [doublearrow, dotted, draw=darkgray, visible on=<{-5,6-8}>] (S\i) to (C\j);
					\draw [doublearrow, draw=violet, visible on=<{5,8}>] (S\i) to (C\j);
				}
				\draw [doublearrow, dotted, draw=darkgray, visible on=<6>] (SN) to [bend left=20] (S\i);
				\draw [doublearrow, draw=green, visible on=<7->] (SN) to [bend left=20] (S\i);
			}

			\begin{scope}[transform canvas={yshift=-1mm}]
				\draw [doublearrow, visible on=<{-3, 4-8}>, dotted, draw=darkgray] (S1) to (S2);
				\draw [doublearrow, visible on=<{-3, 4-8}>, dotted, draw=darkgray] (S2) to (S3);
				\draw [doublearrow, visible on=<{3, 8}>, draw=yellow] (S1) to (S2);
				\draw [doublearrow, visible on=<{3, 8}>, draw=yellow] (S2) to (S3);
			\end{scope}

			\begin{scope}[transform canvas={xshift=3mm}]
				\draw [arrow, visible on=<{-4, 5-8}>, dotted, draw=darkgray] (C3) to node [anchor=west] {\contour{white}{RPC}} (S3);
				\draw [arrow, visible on=<{-4, 5-8}>, dotted, draw=darkgray] (APP) to node [anchor=west] {HTTP} (C3);
				\draw [arrow, visible on=<{4,8}>, draw=red] (C3) to node [anchor=west] {\contour{white}{RPC}} (S3);
				\draw [arrow, visible on=<{4,8}>, draw=red] (APP) to node [anchor=west] {HTTP} (C3);
			\end{scope}
			\begin{scope}[transform canvas={yshift=1mm}]
				\draw [arrow, visible on=<{-4, 5-8}>, dotted, draw=darkgray] (S3) to (S2);
				\draw [arrow, visible on=<{4,8}>, draw=red] (S3) to (S2);
			\end{scope}
		\end{scope}
	\end{tikzpicture}
	\let\thefootnote\relax\footnote{Based on \autocite[Consul Architecture]{ConsulDoc23}}
	\addtocounter{footnote}{-1}

	% Agent (Server/Client), Datacenters
	% Clients are Stateless and Forward RPC calls to Server Agents
	% Servers participate in Raft Quorum. Thus, they save cluster state (like KV keys)
	% Health checks are done though gossip and not consul -> less traffic with consul replication
	% There is no replication of KV store between datacenters, so if a key is stored in a different datacenter, request is forwarded
\end{frame}

% \begin{frame}
% 	\frametitle{Raft Consensus}
% 	\framesubtitle{Short Overview}
% 	\begin{tikzpicture}[remember picture,overlay]
% 		\node[xshift=-1.5cm,yshift=-1.5cm](logo) at (current page.north east){%
% 		\includegraphics[width=3cm]{assets/raft_logo.pdf}};
% 		\node[draw=none,yshift=0.5cm,text width=3cm] at (logo.south) {\baselineskip=1pt\tiny\url{commons.wikimedia.org/w/index.php?curid=105476108}\par};
% 	\end{tikzpicture}

% 	\begin{itemize}
% 		\item<1-> Consensus algorithm for state machine replication\footnote{See \autocite{raft14}}
% 		\item<2-> Not Byzantine Fault Tolerant (leader is trusted)
% 		\item<3-> Crash Tolerance: (N/2) + 1
% 			\begin{itemize}
% 				\item 3 Nodes: 1 Crash
% 				\item 5 Nodes: 2 Crashes
% 				\item 7 Nodes: 3 Crashes
% 			\end{itemize}
% 		% \item<4-> Three states
% 	\end{itemize}
% \end{frame}

\begin{frame}
	\frametitle{Consistency}
	\framesubtitle{in Consul}

	\begin{itemize}
		\item<1-> \emph{Writes}: always sent to leader
		\item<2-> \emph{Reads}: three consistency modes:
			\begin{description}[labelsep=0.5em]
				\setlength\itemsep{0.5em}
				\item[default]<3-> \emph{leader leasing}: leader assumes its role for a specific time window and responds without quorum
				\\
				(However: risk of 2 concurrent leaders $\Rightarrow$ stale reads)
				\item[consistent]<4-> leader has to verify its role before responding
				%				that they are still the leader
				\item[stale]<5-> any server agent (leader \& follower) can respond
			\end{description}
	\end{itemize}
\end{frame}

\subsection{Usage}

\begin{frame}[fragile]
	\frametitle{CLI agent}
	\framesubtitle{for KV store}

	\begin{columns}[c]
		\begin{column}{0.45\textwidth}
						\begin{minted}{bash}
$ consul agent -dev

# other shell:
$ consul kv put my/key 123
$ consul kv get my/key
123
			\end{minted}
		\end{column}

		\begin{column}{0.55\textwidth}
			\onslide<1->{\begin{block}[KV Store]
					\begin{itemize}
						\item<1-> For configuration, locks, metadata, ...
						\item<2-> Max value size of 512 KB
						\item<3-> Requests to KV store via CLI or HTTP API
					\end{itemize}
				\end{block}}
			% Metadata response: https://developer.hashicorp.com/consul/api-docs/kv#metadata-response
		\end{column}
	\end{columns}
\end{frame}

\begin{frame}[fragile]
	\frametitle{Long Polling}
		\begin{minted}{bash}
$ curl -v http://localhost:8500/v1/kv/my/key
# ...
< X-Consul-Index: 19
# ...

$ curl -v http://localhost:8500/v1/kv/my/key?index=19
# blocks until value is changed or timeout is reached
# (max. 10 minutes)
	\end{minted}
\end{frame}


\section{etcd}
\subsection{Overview}
\begin{frame}
	\frametitle{Overview}
	\centering
	% \tikzset{concept/.append style={fill={none}}}
	\vspace{-1cm}
	\scalebox{.7}{\begin{tikzpicture}
			\path [small mindmap, concept color=accent!20, text=black]
			node [concept] {etcd} [clockwise from=0]
			child [concept color=red!20] {
					node [concept] {Use cases} [clockwise from=90]
					child { node [concept] {Coordination} }
					child { node [concept] {Shared configuration} }
					child { node [concept] {Service discovery} }
				}
			child [concept color=green!20] {
					node [concept] {Key/Val store} [clockwise from=-30]
					child { node [concept] {Consistent} }
					child { node [concept] {Distributed} }
					child { node [concept] {Highly available} }
				}
			child [concept color=darkmaroon!20] { node [concept] {gRPC} }
			child [concept color=orange!20, clockwise from=0]{
					node [concept] {Production users} [clockwise from=-160]
					child { node [concept] {Kubernetes} }
					child { node [concept] {Yandex}	}
					child { node [concept] {Huawei} }
				};
		\end{tikzpicture}}
\end{frame}
\begin{frame}
	\frametitle{History}
	\begin{enumerate}
		\item First commit 2013 by CoreOS
		\item 2014 etcd V0.2 - Kubernetes V0.4
		\item 2015 First Stable Release of V2.0
		      \begin{itemize}
			      \item includes Raft
		      \end{itemize}

		\item 2018 CNCF (\textbf{C}loud \textbf{N}ative \textbf{C}omputing \textbf{F}oundation) Incubation
		\item 2019 etcd V3.4
		\item 2021 etcd V3.5
	\end{enumerate}
\end{frame}

\subsection{Architecture}
\begin{frame}
	\frametitle{Architecture}
	\begin{itemize}
		\item Distributed: operates across multiple nodes (cluster)
		\item Consistency ensured by Raft algorithm
		\item leader election if current leader crashes
		\item [\triangleright] high availability, consistency, distribution
	\end{itemize}
\end{frame}
\subsection{Use cases}
\begin{frame}
	\frametitle{Service discovery}
	% TODO Luke
\end{frame}

\begin{frame}
	\frametitle{Distributed coordination}
	\centering
	\begin{tikzpicture}
		\node [roundednode] (ETC) [draw=accent, fill=accent!10] {etcd};

		\node [roundednode] (OP1) [above=of ETC, yshift=.5cm] {Order-processing-1};
		\node [roundednode] (OP0) [left=of OP1] {Order-processing-0};
		\node [roundednode] (OP2) [right=of OP1] {Order-processing-2};
		\node [draw=none] (MS) [above=of OP1] {Micro services};
		\node [roundednode, dashed, node on layer=background, fill=gray!10, draw=darkgray, fit=(OP1)(OP2)(OP0)(MS)] {};

		\node [draw=none, visible on=<2->] (T) [right=of ETC, yshift=-1cm] {order-processing/order123};

		\draw [doublearrow] (OP1) to (ETC);
		\draw [doublearrow] (OP2) to (ETC);
		\draw [doublearrow] (OP0) to (ETC);

		\draw [arrow, dashed, visible on=<2->] (T) to (ETC);
		\begin{scope}[visible on=<3->]
			\draw [arrow, dashed] (OP0) to [bend left=10] (ETC);
			\draw [arrow, dashed] (OP1) to [bend left=20] (ETC);
			\draw [arrow, dashed] (OP2) to [bend right=10] (ETC);
		\end{scope}

		\draw [arrow, dashed, visible on=<4->] (ETC) to [bend left=30] node [sloped, below, midway] {order124} (OP0);
	\end{tikzpicture}
\end{frame}

\begin{frame}
	\frametitle{Configuration management}
	\centering
	\begin{tikzpicture}[node distance=2cm]
		\node [roundednode] (ETC) [draw=accent, fill=accent!10] {etcd};
		\node [roundednode] (IK0) [above=of ETC, yshift=.5cm] {Inventory-keeper-0};
		\draw [arrow, visible on=<{1,2,3,4}>] (IK0) to node[anchor=west, visible on=<1>]{Put \texttt{/cfg/inv/item2/max_capacity 100}} node [anchor=west, visible on=<2-3>]{Watch \texttt{/cfg/inv/item2/max_capacity}} (ETC);
		\node [roundednode] (ADM) [right=of ETC, visible on=<3>, xshift=3.4cm] {Admin};
		\draw [arrow, visible on=<3>] (ADM) to node[anchor=north]{Put \texttt{../max_capacity 200}}(ETC);
		\draw [arrow, visible on=<4>] (ETC) to node[anchor=west, visible on=<4->]{\texttt{/cfg/inv/item2/max_capacity 200 }} (IK0);
	\end{tikzpicture}
\end{frame}

\section{ZooKeeper}
\subsection{Overview}
\begin{frame}
	\frametitle{History}
	\begin{columns}[c]
		\begin{column}{0.75\textwidth}
			\begin{description}
				\item [December 2006] First commit
				\item [November 2007] Version 0.0.1 on Sourceforge
				\item [June 2008] Moved to Apache
				\item [June 2010] First Paper~\cite{10.5555/1855840.1855851}
				\item [November 2010] ZooKeeper becomes a top level project
				\item [Januar 2023] Version 3.8.1
			\end{description}
		\end{column}
		\begin{column}{0.15\textwidth}
			\begin{tikzpicture}
				\sheep[monocle]
			\end{tikzpicture}
		\end{column}
	\end{columns}
\end{frame}

\begin{frame}
	\frametitle{Popularity}
	\framesubtitle{According to Google}
	\centering
	\pgfplotsset{width=\textwidth, height=.7\textheight, compat=1.18}
	\pgfplotstableread[col sep=comma,]{../assets/Zookeeperpopularitygoogle.csv}\datatable
	\tikzstyle{every pin}=[
	rounded corners,
	fill=white,
	draw=black,
	font=\footnotesize,
	]
	\tikzstyle{small dot}=[fill=black,circle,scale=0.3]
	\href{https://trends.google.com/trends/explore?cat=5&date=all&q=Zookeeper}{
		\begin{tikzpicture}
			\begin{axis}[
					axis background/.style={fill=accent!3},
					xlabel=Year,
					ylabel=Search Queries,
					axis line style={rounded corners},
					ytick=\empty,
					xtick={0, 234},
					xticklabels={01/2004, 06/2023},
					x tick label style = {font=\footnotesize, align=center},
					extra x ticks={90, 163.15},
					extra x tick labels={09/2011, 08/2017},
					extra x tick style={grid=major},
				]
				\addplot[color=accent, smooth] table [x=a, y=b, col sep=comma, x expr=\coordindex]{\datatable};
				\node [coordinate, pin=150:{Peak}, small dot] at (axis cs:163.15,81) {};
			\end{axis}
		\end{tikzpicture}}
\end{frame}

\begin{frame}
	\frametitle{What is ZooKeeper?}
	\framesubtitle{ZooKeeper is \ldots}
	\textcolor<2>{gray}{\ldots~a (1)~highly available, (2)~scalable,
		(3)~\textcolor<2>{accent}{distributed}, (4)~\textcolor<2>{accent}{configuration},
		(5)~consensus, (6)~group membership, (7)~leader election, (8)~naming, and
		(9)~coordination \textcolor<2>{accent}{service}}
\end{frame}

\begin{frame}
	\frametitle{What is ZooKeeper again?}
	\framesubtitle{Use cases}
	\centering
	\begin{tcolorbox}[colback=accent!5!white,colframe=accent!75!black,width=.7\linewidth, title=Solve various coordination problems in large distributed system]
		\begin{columns}[c]
			\begin{column}{0.45\textwidth}
				\begin{itemize}
					\item Leader election
					\item Barrier
					\item Queue
					\item Lock
				\end{itemize}
			\end{column}
			\begin{column}{0.45\textwidth}
				\begin{itemize}
					\item Service discovery
					\item Group services
					\item<2> Configuration Stores
				\end{itemize}
			\end{column}
		\end{columns}
	\end{tcolorbox}
\end{frame}

\subsection{Data model}
\begin{frame}
	\frametitle{Data model}
	\begin{columns}[c]
		\begin{column}{0.45\textwidth}
			\tikzstyle{every node}=[roundednode]
			\begin{tikzpicture}[level distance=1cm, sibling distance=1.2cm]
				\node (R) {/}
				child {node (Z) {zoo}
						child {node (D) {cow}}
						child {node (C) {duck}}
					};
				\begin{scope}[visible on=<-2>]
					\node [draw=none] (SR) [right=of R, xshift=1.1cm] {/};
					\node [draw=none] (SZ) [right=of Z, xshift=1cm] {/zoo};
					\node [draw=none] (SD) [below=of SZ.south west, anchor=west, yshift=-1mm] {/zoo/duck};
					\node [draw=none] (SC) [below=of SD.south west, anchor=west, yshift=-2mm] {/zoo/cow};

					\draw [textarrow] (SR) to (R);
					\draw [textarrow] (SZ) to (Z);
					\draw [textarrow] (SD) to (C);
					\draw [textarrow] (SC) to [bend left=20] (D);
				\end{scope}
				\mode<beamer>{
				\begin{scope}[visible on=<3>]
					\node [draw=none] (SZ) [right=of Z, xshift=1cm] {Znodes};

					\draw [textarrow] (SZ) to (R);
					\draw [textarrow] (SZ) to (Z);
					\draw [textarrow] (SZ) to (C);
					\draw [textarrow] (SZ) to [bend left=55] (D);
				\end{scope}
			}
			\end{tikzpicture}
		\end{column}
		\begin{column}{0.45\textwidth}
			\begin{itemize}
				\mode<beamer>{\item \only<1>{File system nodes}
					\only<2->{\textcolor{gray}{\st{File system nodes}} Namespaces}}
				\mode<handout>{\item  Namespaces}
					\item<3-> Three types of Znodes:
					\begin{itemize}
						\item<3-> Persistent
						\item<3-> Ephemeral
						\item<3-> Sequential
					\end{itemize}
				\item<4-> Not designed to store arbitrary blobs
			\end{itemize}
		\end{column}
	\end{columns}
\end{frame}

\begin{frame}
	\frametitle{Operations}
	\begin{columns}[c]
		\begin{column}{0.45\textwidth}
			\onslide<1->{\begin{block}[Znodes]
					\begin{itemize}
						\item create
						\item delete
						\item exists
						\item getChildren
					\end{itemize}
				\end{block}}
		\end{column}
		\begin{column}{0.45\textwidth}
			\onslide<2->{\begin{block}[Data]
					\begin{itemize}
						\item getData
						\item setData
					\end{itemize}
				\end{block}}
			\onslide<3->{\begin{block}
					\begin{itemize}
						\item getAcl
						\item setAcl
						\item sync
					\end{itemize}
				\end{block}}
		\end{column}
	\end{columns}
\end{frame}

\begin{frame}
	\frametitle{Watches}
	\begin{columns}[c]
		\begin{column}{0.25\textwidth}
			\tikzstyle{every node}=[roundednode]
			\begin{tikzpicture}[level distance=1cm, sibling distance=1.2cm]
				\node (R) {/}
				child {node (Z) {zoo}
						child {node (D) {cow}}
						child {node (C) {duck}
								child [visible on=<5->]{node (DD) {Daisy}}
							}
					};
				\duck [scale=.5, xshift=4cm, yshift=-2cm, name=billy, sunglasses=black] (Duck)

				\node [draw=none, visible on=<2>] (B) [right=of R, xshift=1cm, yshift=1cm] {Billy};
				\draw [textarrow, visible on=<2>] (B) to (billy-head);

				\draw [arrow, visible on=<6>] (DD) to [bend right=20] node [draw=none,
				fill=none, sloped, midway, below] {notify} (billy-wing);
				\draw [arrow, visible on=<{4-5,7-}>] (billy-wing) to node [draw=none,
				fill=none,sloped,midway,above] {watch} (C);
			\end{tikzpicture}
		\end{column}
		\begin{column}{0.65\textwidth}
			\begin{sequencediagram}
				\newthread{b}{Client:Billy}
				\newthread{z}{ZooKeeper}
				\newthread{c}{Client:Client 2}
				\begin{call}{b}{watch}{z}{}
				\end{call}
				\begin{call}{c}{add}{z}{}
				\end{call}
				\mess {z}{notify}{b}
				\begin{call}{b}{get}{z}{}
				\end{call}
				\begin{call}{b}{watch}{z}{}
				\end{call}
			\end{sequencediagram}
		\end{column}
	\end{columns}
\end{frame}

\subsection{Architecture}
\begin{frame}
	\contourlength{.09em}
	\frametitle{Architecture}
	\centering
	\begin{tikzpicture}
		\begin{scope}
			\node [roundednode] (S1) [] {Server 1};
			\node [roundednode] (S2) [right=of S1] {Server 2};
			\node [roundednode, draw=accent, fill=accent!3] (S3) [right=of S2] {Server 3};
			\node [roundednode] (S4) [right=of S3] {Server 4};
			\node [roundednode] (S5) [right=of S4] {Server 5};
			\node [draw=none, visible on=<2>] (L) [above=of S2, yshift=1cm] {Leader};
			\draw [textarrow, visible on=<2>] (L) to[bend left=20] (S3);
			\node [draw=none] (ZS) [above=of S3] {\contour*{white}{\textcolor{black}{Ensemble}}};
			\node [roundednode, node on layer=background, dashed, draw=darkgray, fill=gray!10, fit=(S1)(S2)(S3)(S4)(S5)(ZS)] {};

			\begin{scope}[visible on=<5->]
				\draw [arrow] (S1) to [bend left=20] (S3);
				\draw [arrow] (S2) to (S3);
				\draw [arrow] (S4) to (S3);
				\draw [arrow] (S5) to [bend right=20] (S3);
			\end{scope}

			\begin{scope}[visible on=<3->]
				\node [roundednode] (C4) [below=of S3, yshift=-.5cm] {Client 4};
				\node [roundednode] (C5) [right=of C4] {Client 5};
				\node [roundednode] (C6) [right=of C5] {Client 6};
				\node [roundednode] (C7) [right=of C6] {Client 7};
				\node [roundednode] (C3) [left=of C4] {Client 3};
				\node [roundednode] (C2) [left=of C3] {Client 2};
				\node [roundednode] (C1) [left=of C2] {Client 1};
			\end{scope}

			\begin{scope}[visible on=<4->]
				\draw [arrow] (C1) to (S1);
				\draw [arrow] (C2) to (S1);
				\draw [arrow] (C3) to (S2);
				\draw [arrow] (C4) to (S2);
				\draw [arrow] (C5) to (S4);
				\draw [arrow] (C6) to (S5);
				\draw [arrow] (C7) to (S5);
			\end{scope}
		\end{scope}
	\end{tikzpicture}
\end{frame}

\begin{frame}
	\frametitle{Architecture}
	\framesubtitle{An example}
	\centering
	\begin{tikzpicture}
		\begin{scope}
			\node [roundednode] (S1) [] {Server 1};
			\node [roundednode] (S2) [right=of S1] {Server 2};
			\node [roundednode, draw=accent, fill=accent!3] (S3) [right=of S2] {Server 3};
			\node [draw=none] (ZS) [above=of S2, yshift=.5cm] {Ensemble};
			\node [roundednode, node on layer=background, dashed, draw=darkgray, fill=gray!10, fit=(S1)(S2)(S3)(ZS)] {};

			\node [roundednode] (C3) [below=of S3, yshift=-1cm] {Client 3};
			\node [roundednode] (C2) [left=of C3] {Client 2};
			\node [roundednode] (C1) [left=of C2] {Client 1};

			\draw [arrow] (C2) to node [sloped, above, midway, visible on=<2>] {create} (S1);
			\draw [arrow] (S1) to [bend left=20] node [sloped, above, midway, visible on=<3>] {create} (S3);

			\begin{scope}[visible on=<4>]
				\draw [arrow] (S3) to [bend right=10] (S2);
				\draw [arrow] (S3) to [bend right=40] (S1);
			\end{scope}

			\begin{scope}[visible on=<1>]
				\draw [arrow] (C1) to (S1);
				\draw [arrow] (C3) to (S2);
				\draw [arrow] (S2) to (S3);
			\end{scope}
			\begin{scope}[visible on=<2->]
				\draw [arrow, dashed, draw=darkgray] (C1) to (S1);
				\draw [arrow, dashed, draw=darkgray] (C3) to (S2);
				\draw [arrow, dashed, draw=darkgray] (S2) to (S3);
			\end{scope}
		\end{scope}
	\end{tikzpicture}
\end{frame}

\subsection{Configuration management}
\begin{frame}[fragile]
	\frametitle[Absolutely simplified]{Workflow}
	\tikzstyle{every node}=[roundednode]
	\begin{columns}[c]
		\begin{column}{0.65\textwidth}
			\begin{tikzpicture}[level distance=0.8cm, sibling distance=1.4cm, grow'=right, nodes=right]
				\node (R) {/}
				child {node {home}
						child {node {lukas}
								child {node {nvim}
										child {node {expandtab: false}}
										child {node {tabstop: 4}}
									}
							}
						child {node {kai}
								child {node {WebStorm}}
							}
						child {node {tim}
								child {node {ChatGPT}}
								child {node {VSCode}}
							}
					};
			\end{tikzpicture}
		\end{column}
		\begin{column}{0.25\textwidth}
			\begin{onlyenv}<2>
				\begin{minted}{vim}
					set noexpandtab
					set tabstop=4
				\end{minted}
			\end{onlyenv}
		\end{column}
	\end{columns}
\end{frame}


\section{Summary}
\begin{frame}
	\frametitle{Take-home message}
	\begin{itemize}
		\item [\triangleright] If you need \emph{all inclusive} service discovery framework: Consul
		\item [\triangleright] If you need a fast distributed key-value store: etcd
		\item [\triangleright] If you like legacy systems: ZooKeeper
	\end{itemize}
\end{frame}


\section{References}
\begin{frame}[allowframebreaks]
	\frametitle{References}
	\framesubtitle{Consul}
	\printbibliography[keyword={kai}]
\end{frame}
\begin{frame}[allowframebreaks]
	\frametitle{References}
	\framesubtitle{etcd}
	\printbibliography[keyword={tim}]
\end{frame}
\begin{frame}[allowframebreaks]
	\frametitle{References}
	\framesubtitle{ZooKeeper}
	\printbibliography[keyword={lukas}]
\end{frame}
\begin{frame}[allowframebreaks]
	\frametitle{References}
	\framesubtitle{General}
	\printbibliography[keyword={gen}]
\end{frame}

\end{document}